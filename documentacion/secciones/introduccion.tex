\section{\Large Introducción}
En los sistemas celulares, el área de cobertura se divide en celdas, típicamente representadas como hexágonos regulares. Cada celda contiene una estación base que proporciona servicio a los usuarios dentro de su área. Dado que el espectro de frecuencias es limitado, se reutilizan frecuencias en celdas alejadas según un factor de reuso k.

Los sistemas celulares modernos son la base de las redes de telecomunicaciones móviles. Su arquitectura se basa en una división geográfica del área de cobertura en regiones llamadas \textbf{celdas}, cada una de las cuales es atendida por una \textbf{estación base (BS)}.

La forma hexagonal es la más utilizada en modelos teóricos por su eficiencia geométrica, permitiendo una cobertura continua y uniforme sin traslapes ni huecos.

Dado que el espectro radioeléctrico es limitado, las frecuencias deben reutilizarse en celdas no adyacentes. Este mecanismo de reutilización está regido por el \textbf{factor de reuso} \( k \), que determina la distancia mínima necesaria entre dos celdas que usan el mismo canal de frecuencia.

\subsection*{Modelo Lognormal de Propagación}

Este modelo considera que la pérdida total \( L \) en decibelios (dB) está compuesta por una pérdida determinística debido a la distancia y una componente aleatoria gaussiana:

\[
L = L_0 + 10 \alpha \log_{10} \left( \frac{d}{d_0} \right) + X_\sigma
\]

Donde:
\begin{itemize}
    \item \( L_0 \): pérdida a distancia de referencia \( d_0 \)
    \item \( \alpha \): es el exponente de pérdida que caracteriza el entorno. En zonas urbanas, su valor comúnmente oscila entre 2.5 y 4.0. (en esta práctica, \( \alpha = 2.8 \))
    \item \( d \): distancia entre la BS y el usuario
    \item \( X_\sigma \): componente aleatoria (ensombrecimiento), con \( \sigma = 7 \) dB
\end{itemize}

Este modelo captura de manera realista las variaciones espaciales de la señal, permitiendo simular escenarios más cercanos a una red real.

\subsection*{Asociación de usuarios a Estación Base}

Dado un conjunto de usuarios distribuidos aleatoriamente en el área de cobertura, el proceso de asociación consiste en asignar a cada usuario la BS con la menor pérdida total. Es decir, para un usuario \( u \), se evalúa su pérdida hacia cada BS \( i \in \{0, 1, \dots, 6\} \) y se selecciona aquella que minimiza:  
\[
j = \arg\min_{i} L_{i,u}
\]

Este criterio garantiza la mejor calidad de señal posible para el usuario dentro de las celdas consideradas.

\subsection*{Señal a interferencia (SIR) y tasa de transmisión}
Una vez realizada la asociación, es importante evaluar la calidad de la comunicación en términos de la relación entre la potencia de la señal deseada y la potencia de las señales interferentes provenientes de otras BS que utilizan la misma frecuencia. Esta relación se conoce como SIR (Signal-to-Interference Ratio) y se calcula como:

\[
\text{SIR}_u = \frac{P_r^{(\text{serv})}}{\sum_{i=1}^{N} P_r^{(\text{interf})}}
\]

Donde: 
\begin{itemize}
    \item $P_r^{(\text{interf})}$ son las potencias recibidas desde las BS interferentes
    \item $P_r^{(\text{serv})}$ es la potencia recibida desde la BS que atiende al usuario
\end{itemize}

El valor del SIR impacta directamente en la capacidad de transmisión. Utilizando la fórmula de Shannon, se puede estimar la tasa de transmisión máxima teórica que puede alcanzar un canal:

\[
C = B \cdot \log_2(1 + \text{SIR})
\]

donde \( C \) se mide en bits por segundo (bps), y \( B \) es el ancho de banda del canal en Hz.

Este conjunto de herramientas permite modelar el comportamiento de una red celular en términos de cobertura, interferencia y capacidad, proporcionando una base sólida para el análisis y diseño de sistemas de telecomunicaciones.