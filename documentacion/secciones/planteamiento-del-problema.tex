\section{Planteamiento del Problema}
Los proveedores comerciales de antenas suelen entregar información parcial de las características de radiación, proporcionando únicamente cortes bidimensionales (patrones 2D) o datos aislados, sin el modelo 3D completo. Esto limita la capacidad de ingenieros e investigadores para visualizar y analizar el comportamiento real de la antena en el espacio tridimensional, especialmente cuando no se dispone de software comercial avanzado como HFSS. Por otro lado, algunos fabricantes aseguran que sí proporcionan toda la información necesaria para reconstruir el patrón 3D, lo que genera ambigüedad sobre la disponibilidad y consistencia de los datos. 
\\
Asimismo, el acceso rápido a un modelo 3D de la antena puede ser de gran utilidad en entornos académicos, de investigación y en proyectos de validación, donde no siempre se cuenta con licencias de herramientas costosas ni los recursos de cómputo adecuados. De esta manera, surge la necesidad de contar con una de fácil acceso que permita generar de manera dinámica el patrón de radiación 3D de una antena a partir de: 
\begin{itemize}
  \item Información parcial (por ejemplo, cortes 2D en los planos E y H), 
  \item La ecuación de propagación completa que define matemáticamente el comportamiento de la antena en el espacio lejano.
\end{itemize}

Esta investigación cuantitativa se propone resolver la siguiente pregunta:
\begin{quotation}
  \textit{\\¿Cómo diseñar e implementar una aplicación web que genere patrones de radiación tridimensionales de manera precisa, a partir de información parcial o de la ecuación completa de propagación, sin recurrir a modelado físico ni software comercial?}
\end{quotation}