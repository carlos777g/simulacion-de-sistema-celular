\section{\Large Desarrollo}

\subsection*{Parte 1: Asociación a BS}
\begin{enumerate}
    \item Generar una red de 7 celdas hexagonales: una central (BS 0) y su primer anillo de interferencia (BS 1 a 6).
    \item Distribuir 10 usuarios aleatorios dentro de la región.
    \item Calcular las pérdidas totales usando el modelo lognormal.
    \item Asignar cada usuario a la BS con menor pérdida.
    \item Graficar la asociación y exportar los datos de los usuarios asociados a la BS 0.
\end{enumerate}

\subsection*{Parte 2: Cálculo de SIR y Tasa de transmisión}
\begin{enumerate}
    \item Identificar usuarios atendidos por la BS 0.
    \item Calcular las potencias recibidas desde BS 0 (señal) y las interferentes.
    \item Calcular el SIR y la tasa de transmisión de cada usuario.
\end{enumerate}
